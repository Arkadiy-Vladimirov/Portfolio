\subsubsection{Абелева группа}
    Итак, пусть некоторая группа $A$ --- абелева. Известно, что для 
    \emph{конечно-порожденных} абелевых групп справедливо разложение 
    (\cite{Vinberg} гл.9 \S 1)
    \begin{equation}\label{A_decomp}
        A = \sub{A} \oplus \Tor A,
    \end{equation}
    где $\sub{A} \simeq \mathbb{Z}^{n}$ --- \emph{свободная подгруппа}, 
    $\Tor A$ --- \emph{подгруппа кручения}, т.е.
    \begin{equation}\label{tor_def}
        \Tor A \doteqdot \{a \in A: ma = 0\text{ для некоторого }m \in 
        \mathbb{Z}, m \ne 0\}.
    \end{equation}
    
    Из разложения \eqref{A_decomp} и леммы \ref{lm_char_decomp} следует, что
    \begin{equation}
        X(A) \simeq X(\sub{A}) \oplus X(\Tor A).
    \end{equation}
    Но из определения группы кручения \eqref{tor_def} и свойств характера 
    вытекает, что $X(\Tor A)$ тривиально. Таким образом, получим
    
    \begin{statement} \label{st_ab_char} Для конечно-порожденной абелевой группы
        \[X(A) \simeq X(\sub{A}) \simeq \mathbb{C}^{m},\]
    где $m = \dim \sub{A}$.
    \end{statement}