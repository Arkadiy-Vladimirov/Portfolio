\subsubsection{Группоид}
    Нам потребуется следующая очевидная

    \begin{lemma}\label{lm_char_decomp}
        Для любых двух группоидов $\Gamma_1$ и $\Gamma_2$ справедливо
        \[X(\Gamma_1 \times \Gamma_2) \simeq X(\Gamma_1) \oplus X(\Gamma_2).\]
    \end{lemma}
    \begin{proof}
        В самом деле, для любого $\chi : \Gamma_1 
        \times \Gamma_2 \to \mathbb{C}$, существуют единственные
        $\chi_1 : \Gamma_1 \to \mathbb{C}$, $\chi_2 : \Gamma_2 \to \mathbb{C}$ 
        такие, что диаграмма (рис.~\ref{cd_char_sum}) коммутативна.

        \begin{figure}[h]
            \centering
            \[\xymatrix{
                \Gamma_1 \times \Gamma_2 \ar[rr]^{\textstyle \chi_b \times \chi_c} \ar[dd]_{\textstyle \chi_a}  & & \mathbb{C} \times \mathbb{C} \ar[lldd]^{\textstyle +}\\
                                                                                                                & &                                                      \\
                \mathbb{C}                                                                                      & &
            }\]
            \caption{}
            \label{cd_char_sum}
        \end{figure}
    \end{proof}

    Доказанная лемма вместе с утверждением \ref{st_groupoid_decomp} 
    дают важное
    \begin{statement}[о разложении характера группоида]\label{st_char_grpd_decomp}
        \[X(\Gamma) \simeq X(\Gamma/\Phi_\Gamma) \oplus X(\Phi_\Gamma).\]
    \end{statement}

    Которое позволяет нам вместо рассмотрения характера на группоиде целиком,
    отдельно изучить характеры \emph{простого группоида} ($\Gamma/\Phi_\Gamma$) и
    группы ($\Phi_\Gamma$).

    Первый случай достаточно тривиален. Действительно, как уже было показано 
    (следствие \ref{cor_simple_grp}), все стрелки простого 
    группоида можно однозначно разложить\\ $v = gf^{-1}$ по некоторому вееру $V$,
    что ввиду свойств характера дает его выражение через значения на базисе 
    $\chi(v) = \chi(g) - \chi(f)$.

    Отсюда ясно, что характер простого группоида 
    однозначно определен $n-1$\footnote{значение на тождественной стрелке 
    автоматически задано нулем} числом~--- его значениями на стрелках некоторого 
    веера. Иначе говоря справедливо

    \begin{statement}\label{st_smp_grp_char} Для простого группоида $\Gamma$
        \[X(\Gamma) \simeq \mathbb{C}^{n-1},\]
        где $n$~--- число объектов $\Gamma$.
    \end{statement}

    Случай группы мы обудим в следующих параграфах.
