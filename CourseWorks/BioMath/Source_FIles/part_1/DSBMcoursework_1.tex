\documentclass[oneside,final,12pt]{article}
\usepackage[utf8]{inputenc}
\usepackage[russianb]{babel}
\usepackage{vmargin}
\usepackage{amsmath}
\usepackage{amsfonts}
\usepackage{amssymb}
\usepackage{amsthm}
\usepackage[matrix,arrow,curve]{xy}
\usepackage{graphicx}
\usepackage{subfig}
\setpapersize{A4}
\setmarginsrb{2cm}{1.5cm}{1cm}{1.5cm}{0pt}{0mm}{0pt}{13mm}
\usepackage{indentfirst}
\sloppy

\newcommand*\Rn  [1]{\mathbb{R}^{#1}}
\renewcommand*\leq{\leqslant}
\newcommand*\Za {\mathbb{Z}}
\newcommand*\Na {\mathbb{N}}
\renewcommand*\equiv{\Leftrightarrow}
\newcommand*\eps{\varepsilon}

\newcommand*\vat[2]{\left.#1\right|_{#2}}
\newcommand*\abs[1]{|#1|}
\newcommand*\norm[2]{\|#1\|_{#2}}
\newcommand*\segm[2]{[#1,#2]}
\newcommand*\inter[2]{(#1,#2)}
\newcommand*\scprod[2]{\bigl\langle #1 , #2 \bigl\rangle}

\newcommand*\picsize{0.5\textwidth}
\newcommand*\bigpicsize{0.7\textwidth}
\newcommand*\subpicsize{0.45\textwidth}
\newcommand*\subtripicsize{0.3\textwidth}
\newcommand*\picpath{pictures/}

\theoremstyle{plain}
\newtheorem*{theorem}{Теорема}
\theoremstyle{remark}
\newtheorem*{remark}{Замечание}
\theoremstyle{definition}
\newtheorem{definition}{Определение}
\theoremstyle{plain}
\newtheorem{statement}{Утверждение}

\begin{document}
	\begin{titlepage}
		\begin{centering}
			\includegraphics[width=0.5\textwidth]{\picpath msu.png}\\
			{\scshape Московский государственный университет имени М.~В.~Ломоносова}\\
			Факультет вычислительной математики и кибернетики\\
			Кафедра системного анализа\\
			\vfill
			{\LARGE Курсовая работа. Часть 1}\\
			\vspace{1cm}
			{\Huge\bfseries "<Исследование динамических систем с дискретным временем">\\}
		\end{centering}
		\vspace{1cm}
		\begin{flushright}
			\begin{large}
				{\itshape Студент 315 группы\\}
				А.~А.~Владимиров\\
				\vspace{5mm}
				{\itshape Научный руководитель\\}
				Д.~А.~Алимов\\
			\end{large}
		\end{flushright}
		\vfill
		\begin{centering}
			Москва, 2021\\ 
		\end{centering}
	\newpage
	\end{titlepage}
	\setcounter{page}{2}
	
	\tableofcontents

	\section{Постановка задачи}\label{sec_task}
		
		Перед нами одномерная дискретная динамическая система
	\begin{equation} \label{ds1}
		u_{t+1} = r \sqrt{u_t} (1-u_t), \quad 0 < u_t < 1, \: t \in \Na,
	\end{equation}
		и дискретная динамическая система с запаздыванием
	\begin{equation} \label{ds2}
		u_{t+1} = r \sqrt{u_t} (1-u_{t-1}), \quad 0 < u_t < 1, \: t \in \Na.
	\end{equation}
		Требуется провести качественный анализ предложенных систем.

	\bigskip
		Для удобства дальнейшего изложения отметим, что система \eqref{ds1} и, с некоторыми оговорками (о которых позже), система \eqref{ds2} являются частными случаями такого объекта, как \emph{однопараметрическая дискретная динамическая система}, определяемая отображением \(f\):
		\begin{equation}	\label{dds}
			u \mapsto f(u) = f(u;r), \quad u \in U \subset \Rn{n}, \: r \in \Rn{}, \quad f : U \rightarrow U.
		\end{equation}
		Здесь множество \(U\) -- фазовое пространство, \(r\) -- параметр.

	\section{Анализ первой системы}
		
		Для системы \eqref{ds1} отображение \(f : U \rightarrow U\)  имеет вид:
		\begin{equation}\label{ds1func}
			 f(u;r) = r \sqrt{u} (1-u), \quad U = \inter{0}{1}, \: r \in \Rn{}.
		\end{equation}
		Перейдем к её анализу.

	\subsection{Неподвижные точки и их устойчивость}
		В случае дискретной динамической системы элементарно вводится
		\begin{definition}[\cite{DSMB}]\label{stp_def}
			\emph{Неподвижными точками} системы \eqref{dds} называются такие точки фазового пространства \(u^* \in U\), что \( f(u^*) = u^* \). 
		\end{definition}
		Таким образом, чтобы найти все неподвижиные точки системы \eqref{ds1} достаточно решить уравнение
		\[ r \sqrt{u} (1-u) = u, \text{ на интервале } \inter{0}{1},\]
		что мы и сделаем. В условиях \( u \in \inter{0}{1} \) справедливо
		\begin{gather*}
			r \sqrt{u} (1-u) = u \equiv \\
			\equiv r (1-u) = \sqrt{u}.
		\end{gather*}
		Заметим, что левая часть уравения принадлежит интервалу \(\inter{0}{r}\), правая ~--- интервалу \(\inter{0}{1}\), следовательно решения существуют лишь при \( r \in \inter{0}{1} \). Наложив это допольнительное ограничение имеем
		\begin{gather*}
			r (1-u) = \sqrt{u} \equiv  \\
			\equiv r^2 u^2 - (1+2r^2) u + r^2 = 0.
		\end{gather*}
		Полученное квадратное уравнение имеет решения
		\[ u_{1,2} = \frac{1+2r^2 \pm \sqrt{1+4r^2}}{2r^2}.\]
		Нетрудно проверить, что наложенным нами ограничениям удовлетворяет лишь решение
\newcommand*{\ep}{\frac{1+2r^2 - \sqrt{1+4r^2}}{2r^2}}
		\begin{equation} \label{equ_pnt}
			 u^* = \ep, 
		\end{equation}
		причем для всех \(r\) из \(\inter{0}{1}\).
		
		\bigskip \bigskip \bigskip

		Перейдем к исследованию вопроса устойчивости единственной неподвижной точки \(u^*\) \eqref{equ_pnt}, существующей при значениях параметра \( r \in \inter{0}{1} \).
		\begin{definition}[\cite{DSMB}]
			Неподвижная точка \(u^*\) системы \eqref{dds} называется \emph{устойчивой по Ляпунову}, если для любого \(\eps > 0\) существует такое \(\delta > 0\), что для любых начальных данных \(u_0\) из \(\delta\)-окрестности точки \(u^*\) вся траектория системы \(u_t, \: t = 0,1,2\ldots\) содержится в \(\eps\)-окрестности точки \(u^*\).

			Если, кроме того, \[\lim_{t \to \infty} f(u_t) = u^*,\] то точка \(u^*\) называется \emph{асимптотически устойчивой.}
		\end{definition}
		В дальнейшем нам пригодится следующее полезное
		\begin{statement}[\cite{DSMB}] \label{st1}
			Пусть \(u^*\)  ~--- неподвижная точка одномерной системы \eqref{dds}, и пусть \(f\) обратима в малой окрестности \(u^*\). Тогда \(u^*\) асимптотически устойчива, если \( \abs{f_u(u^*)} < 1 \), и неустойчива, если \( \abs{f_u(u^*)} > 1 \).
		\end{statement}
		Таким образом, "<в первом приближении"> вопрос об устойчивости неподвижной cводится к рассмотрению её мультипликатора\footnote{Величину \(\abs{f_u(u^*)}\) называют \emph{собственным значением} или \emph{мультипликатором} неподвижной точки одномерной динамической системы с дискретным временем (см. \cite{DSMB} п.3.4, стр. 77).}.
		
		Итак, 
		\begin{equation} \label{mult1}
			f_u(u^*;r) = \left( r\sqrt{u^*}(1-u^*) \right)_u = r \left( \frac{1}{2\sqrt{u^*}} - \frac{3}{2}\sqrt{u^*} \right) =  r \left( \frac{1-3u^*}{2\sqrt{u^*}} \right) = r \left( \frac{1-3\ep}{2\sqrt{\ep}} \right).
		\end{equation}
		Довольно тяжелый анализ выражения \eqref{mult1} предоствим компьютеру. Тем не менее, далее нам понадобятся значения  \(r \in \inter{0}{1}\), на которых фунция \( \mu(r) = f_u(u^*;r)\) обращается в нуль, которые мы вычислим аналитически.
		\begin{multline*}
			\mu(r) = 0 \equiv f_u(u^*;r) = 0 \equiv r \left( \frac{1-3u^*}{2\sqrt{u^*}} \right) = 0 \equiv u^* = \frac{1}{3} \equiv \ep = \frac{1}{3} \equiv \\ \equiv 3 + 6r^2 - 3\sqrt{1+4r^2} = 2r^2 \equiv 3 + 4r^2 = 3\sqrt{1+4r^2} \equiv  9+24r^2 + 16r^4 = 9 + 36r^2 \equiv \\ \equiv 16r^4 - 12r^2 = 0 \equiv r^2(4r^2-3) = 0 \equiv r = \frac{\sqrt{3}}{2}.
		\end{multline*}
		Единственное полчученное значение обозначим за
		\begin{equation} \label{r_0}
			r_0 = \frac{\sqrt{3}}{2}.
		\end{equation}

		Наконец, приведём график мультипликатора неподвижной точки \(u^*,\: \mu(r) = f_u(u^*;r) \), построенный в среде MatLab\footnote{Подробнее со всеми вычислениями проведёнными в MatLab и использованными в работе вы можете ознакомится в приложенном .mlx файле.}.
		\begin{figure}[h]
			\centering
			\includegraphics[width=\picsize]{\picpath fig1}
			\caption{Зависимость величины мультипликатора от параметра.}
		\end{figure}

		Так
		 \[\begin{array}{lcl}
			\mu(r) \in \inter{0}{1}, &\text{при}& r \in \inter{0}{r_0},\\
			\mu(r) = 0, &\text{при}& r = r_0,\\
			\mu(r) \in \inter{-1}{0}, &\text{при}& r \in \inter{r_0}{1},\\
		\end{array}\]
		что, как следует из утв. \ref{st1}, говорит об асимптотической устойчивости точки \(u^*\).

		\bigskip
		Подведём итог. При значениях параметра \(0<r<1\) в системе \eqref{ds1} существует одна неподвижная точка \(u^* = \ep\), при всех прочих \(r\)  неподвижных точек нет. В случае существования неподвижной точки, она является асимптотически устойчивой для всех \(0<r<1\). Мультипликатор точки \(u^*\) знакопеременен и обращается в нуль при \(r_0 = \frac{\sqrt{3}}{2} \), а потому устойчивость точки имеет разный характер в зависимости от значения \(r\). При \(r \in (0,r_0]\) \(u^*\) неподвижная точка устойчива монотонно, когда же \(r \in \inter{r_0}{1} \), близкие орбиты сходятся к неподвижной точке колебательным образом \footnote{см. \cite{DSMB} п.3.4, стр.77.}. Этот эффект можно наблюдать на рис.\ref{fig1}.

	\begin{figure}[!h]
		\centering
		\subfloat[\centering Монотонный (\(r<r_0\)).] 
			{{\includegraphics[width=\subpicsize]{\picpath ep_mon.pdf} }}
		\qquad
		\subfloat[\centering Колебательный (\(r>r_0\)).]
			{{\includegraphics[width=\subpicsize]{\picpath ep_osc.pdf} }}
		\caption{Харктер сходимости к неподвижной точке.} \label{fig1}
	\end{figure}

	Теперь, считая вопрос о неподвижных точках в достаточной степени исчерпанным, мы можем перейти к исследованию системы \eqref{ds1} на наличие циклов, их длину и устойчивость.
		
	\subsection{Циклы}

	\begin{definition}[\cite{DSMB}]\label{cycle}
		Циклом длины \(k\) дискретной динамической системы
		\[u_{t+1} = f(u_t), \quad u_t \in \Rn{},\]
		называется множество различных точек \(u_1, u_1, \ldots, u_k \) таких, что 
		\[u_2 = f(u_1), \ldots, u_k = f(u_{k-1}), u_1 = f(u_k).\] 
	\end{definition}
	\begin{remark}[\cite{DSMB}]\label{stable}
		В силу определения цикла, каждая из точек \(u_i, i = 1,2,\ldots,k\), является неподвижной точкой \(k\)-ой итерации отображения 
		\[f^k(u) = \underbrace{f \circ \ldots \circ f}_{k}.\]
		Таким образом, вопрос об устойчивости цикла сводится к вопросу об устойчивости неподвижных точек отображения \(f^k\), которые состовляют цикл длины \(k\).
	\end{remark}
	
	Итак, для нахождения циклов длины \(k\) нам потребуется найти \(k\) различных решений задачи
	\begin{equation} f^k(u) = u, \label{cycle_eq1} \end{equation}
	удовлетворяющих определению \ref{cycle}. В общем случае это довольно тяжелая задача, поэтому решение ищут численно. Как известно, любой численный алгоритм "<адекватно"> работает на корретно поставленных, а значит устойчивых задачах. Следовательно, на нашу задачу придется наложить еще одно условие
	\begin{equation} \frac{df^k(u,r)}{du} < 1,\label{cycle_eq2} \end{equation}
которое (см. зам. \ref{stable}) гарантирует устойчивость искомого цикла. Так, наша задача сводится к нахождению устойчивых циклов \eqref{cycle_eq1}, \eqref{cycle_eq2}. Решим её для \(k = 2,3\).

	\bigskip
	Перед поиском циклов сделаем ещё пару наблюдений. Первое ~--- до сих пор, ввиду ненадобности мы не уточняли область \(R\) допустимых параметров \(r\),  при которых динамическая система на очередной итерации не может покинуть фазовое пространство \(U = \inter{0}{1} \), а значит остаётся корректно заданной. Нетрудно показать, что область \(R\) представляет из себя интервал \(\left(0,\frac{3\sqrt{3}}{2}\right)\).

	Для этого рассмотрим функцию 
	\[g(u) = \sqrt{u}(1-u), \quad u \in \inter{0}{1}.\]
	Её производная равна
	\[g'(u) = \frac{1}{2\sqrt{u}} - \frac{3}{2}\sqrt{u},\]
	И обращатся в ноль лишь при \(u = \frac{1}{3}, \: u \in \inter{0}{1}\). Более того, вторая производная функции
	\[g''(u) = -\frac{1}{4\sqrt{u}}\left(\frac{1}{u}+3\right) < 0\] 
на всём интервале \(\inter{0}{1}\). Отсюда, очевидно, следует, что в точке \(u = \frac{1}{3}\) фунция \(g(u)\) достигает максимума на \(\inter{0}{1}\), и
	\[0<g(u)<g\left(\frac{1}{3}\right) = \frac{2}{3\sqrt{3}}, \quad u \in U.\]

 	Как нетрудно заметить, функция \(f(u;r) = rg(u)\), а значит, для того, чтобы \(\forall u \in U\), выполнялось \(f(u;r) \in U\), необходимо и достаточно, чтобы \(r \in R = \left(0,r_{max}\right), \text{ где } r_{max} = \frac{3\sqrt{3}}{2}\).
	
	\bigskip
	Теперь, второе наблюдение. Функция \(f(u;r)\), определяющая динамическую ситему \eqref{ds1} непрерывно зависит от параметра \(r\)\footnote{более того, \(f(u;r) \in C^{\infty}(U,R)\).}, что позволяет нам ожидать непрерывной зависимости траекторий системы от параметра \footnote{Доказательство такого смелого утверждения выходит за рамки курса \cite{KURS}, впрочем, автор предполагает его  справедливость, так как иначе многие техники, используемые в курсе, и, как следствие вынужденные быть использованными  в данной работе, были бы попросту неправомерны.}. Далее, производная по \(u\) отображения  \(f^k(u;r)\), характеризующая \eqref{stable} устойчивость цикла (в случае, разумеется его существования) по теореме о производной сложной функции равна 
	\[ f^k(u;r)'_u = \underbrace{f_u'(u;r) \cdot \ldots \cdot f_u'(u;r)}_{k} = (f_u'(u;r))^k \stackrel{\eqref{mult1}}{=} r^k \left( \frac{1}{2\sqrt{u}} - \frac{3}{2}\sqrt{u} \right)^k = \frac{r^k}{2^k u^{\frac{k}{2}}}(1-3u)^k,\]
	и, следовательно, является монотонно возрастающей по \(r\) на \(R\). Это, в частности, означает, что локально \emph{свойство устойчивости сохраняется при уменьшении \(r\)}.

	Наконец, учитывая вышесказанное, для нахождения усточивых циклов, вместо решения задачи \eqref{cycle_eq1} с "<тяжёлым"> ограничением \eqref{cycle_eq2}, мы можем решить систему в "<пограничном"> случае
	\begin{equation} \begin{array}{rcl} \label{cycle_eq_sys}
		f^k(u;r) &=& u,\\
		f^k(u;r)'_u &=& 1;\\
	\end{array} \end{equation}
и немного уменьшая \(r\) ожидать сохранения существования устойчивого решения. Следуя такой логике, получим \(k\) различных решений, подозрительных на образование цикла, выберём наименьшее \(r_0\), перерешаем уравнение 
	\[f^k(u;r_0) = u,\]
и, наконец, вероятно получим устойчивый цикл длины \(k\). Так мы и поступим.

	\bigskip
	Следуя обсуждённому алгоритму, для \(k = 2\)  получим набор решений
	\[\begin{pmatrix} u_1 \\ r_1\end{pmatrix} \approx \begin{pmatrix} 0,284\\ 2,268\end{pmatrix}, \begin{pmatrix} u_2 \\ r_2\end{pmatrix} \approx \begin{pmatrix} 0,398 \\ 2,056 \end{pmatrix}, \begin{pmatrix} u_3 \\ r_3 \end{pmatrix} \approx \begin{pmatrix} 0,449\\ 1,218\end{pmatrix}.\]
	И действительно при \( r_0 = 2 < r_1, r_2\) получим устойчивый цикл длины 2 в точках
	\[\hat u_1 \approx 0,451, \hat u_2 \approx 0,737,\]
	что и иллюстрирует	рис. \ref{fig3}.

	\begin{figure}[!h]
			\centering
			\includegraphics[width=\picsize]{\picpath fig3}
			\caption{Устойчивый цикл длины 2 (\(r = 2, u_0 = 0,5\)).} \label{fig3}
	\end{figure}
	
	Подобной ситуации в случае \(k = 3\) не наблюдается. Даже достаточно плотная выборка начальных значений и высокая точность вычислительного алгоритма не даёт нужных результатов. Поэтому, с достаточной степенью уверенности можно утверждать что устойчивого цикла длины \(3\)  системe \(\eqref{ds1}\) не возникает. В качестве иллюстрации происходящего при увеличении \(r\) приводём график на рис. \ref{fig4}.

	\begin{figure}[!h]
			\centering
			\includegraphics[width=\picsize]{\picpath fig4}
			\caption{Потеря устойчивости (\(r = 2,5,  u_0 = 0,5\)).} \label{fig4}
	\end{figure}

	Помимо прочего, об отсутствии устойчивого цикла длины \(3\) можно судить по бифуркационной диаграмме системы \eqref{ds1}, представляющей собой классический каскад удвоения периода. На риc. \ref{fig5} изображены последовательные бифуркации системы \eqref{ds1}, полученные с помощью компьютерного итерирования отображения \eqref{ds1func}.

	\begin{figure}[!h]
			\centering
			\includegraphics[width=\picsize]{\picpath fig5.png}
			\caption{Каскад удвоения периода в дискретной динамической системе \eqref{ds1}. \(r \in \inter{0}{r_{max}}\).} \label{fig5}
	\end{figure}

	В дополнение, исследуем динамику \emph{показателя Ляпунова} некоторой траектории, что даст некоторое представление об устойчивости системы и изменении этого свойства в зависимости от параметра\footnote{Подробнее см.\cite{DSMB}, п.3.6, стр. 86.}.
	\begin{definition}[\cite{DSMB}]
		Пусть \(f:\Rn{} \rightarrow \Rn{}\) ~--- гладкое отображение. \emph{Показателем Ляпунова} траектории \(u_1,u_2,\ldots,u_n,\ldots\) называется величина
		\[h(u_1) = \lim_{n\rightarrow\infty} \frac{\ln\abs{f'(u_1)} + \ln\abs{f'(u_2)} + \ldots + \ln\abs{f'(u_n)}}{n}, \qquad (\ln0 := -\infty)\]
если этот предел существует.
	\end{definition}

	Следующий график (рис. \ref{lyap1}) демонстрирует зависимость показателя Ляпунова траекторий, выпущенных из точек сетки \texttt{u1 = 0:0.1:1}, от параметра \(r\).

	\begin{figure}[!h]
			\centering
			\includegraphics[width=\picsize]{\picpath lyap1}
			\caption{Показатель Ляпунова. \(u_0 = 0,5, r \in \inter{0}{r_{max}}\).} \label{lyap1}
	\end{figure}
Как видно из графика, распределение показателя Ляпунова не зависит от начальной точки траектории (графики, построенные в разных точках сетки наложились друг на друга), что позволяет в данном конкретном случае говорить о показателе Ляпунова, как о \emph{характеристике системы в целом}. Также отметим, что в окрестности \(r_{max}\) показатель Ляпунова становиться положительным, т.е. система становится неустойчивой и, более того, хаотичной, что подтверждается бифуркационной диаграммой (рис. \ref{fig5}) и соответствует нашим представлениям о смысле показателя Ляпунова\footnote{см. там же.}.

	Итак, на определенном промежутке значений параметра в системе \eqref{ds1} наблюдается возникновение устойчивого цикла длины \(2\) (рис. \ref{fig3}). Судя по бифуркационной диаграмме (рис. \ref{fig5}) и динамике показателя Ляпунова (рис. \ref{lyap1}), при дальнейшем увеличении параметра можно предположить появление устойчивых циклов большего периода (но не периода \(3\) (рис.\ref{fig4})), после чего, поведение системы становится хаотическим. 

	На этом мы заканчиваем исследование системы \eqref{ds1} и переходим к системe \eqref{ds2}.

 	\section{Анализ второй системы}
		Для удобства, повторно приведем вид рассматриваемой системы \eqref{ds2} 
			\[u_{t+1} = r \sqrt{u_t} (1-u_{t-1}), \quad 0 < u_t < 1, \: t \in \Na,\]
которая, как было отмечено в пункте \ref{sec_task}, является частным случаем системы вида \eqref{dds}. Действительно, тривиальной заменой переменных
			\begin{equation}\label{ds2subst}\begin{aligned}
				&u_1(t+1) = r \sqrt{u_1(t)} (1-u_2(t)),\\
				&u_2(t+1) = u_1(t),
			\end{aligned}\end{equation}
система с запаздыванием \eqref{ds2} приводится к обыкновенной двумерной динамической системе \eqref{dds}.

		Впрочем, некоторые особенности у многомерной дискретной динамической системы, сводимой к системе с запаздыванием всё же имеются. Так, например, неподвижные точки такой системы являются решениями уравнения
		\begin{equation}\label{stp_def2}
			u^* = f(u^*,\ldots,u^*), \quad u^* \in \Rn{},
		\end{equation}
где \(f(u_t, u_{t-1}, \ldots, u_{t-T})\) ~--- функция задающая систему с запаздыванием\footnote{см. \cite{DSMB} п.9.4, стр.224-225.}.
		
	Перед тем как перейти к анализу системы, напомним, что фазовое пространство имеет вид \(U = \inter{0}{1}\) в записи \eqref{ds2}, и вид \(U = \inter{0}{1} \times \inter{0}{1}\) в записи \eqref{ds2subst}. Также, вид \eqref{ds2} позволяет \emph{при первом рассмотрении}\footnote{О том, почему именно при \emph{первом рассмотрении} ~--- позже.} полагать, что, аналогично \eqref{ds1}, область допустимых параметров предстваляет из себя интервал \(R = \left(0,r_{max}\right), \text{ где } r_{max} = \frac{3\sqrt{3}}{2}\).

	\subsection{Неподвижные точки}\label{section_stp}
		Уравнение \eqref{stp_def2} для системы \eqref{ds2} имеет вид
			\[u^* = r \sqrt{u^*} (1-u^*), \quad u^* \in U,\: r \in R.\]
		Анализ \footnote{При исследовании систем 2 и 3, c целью сокращения технических выкладок в тексте работы, б\'{о}льшая часть аналитических вычислений произведена посредством пакета символьных вычислений MatLab. Ознакомится с ними можно, по прежнему, в прикрепленных .mlx файлах.} решений этого уравнения, позволяет сделать вывод о существовании единственной неподвижной точки 
		\begin{equation}\label{stp_ds2}
			u^* = \frac{(\sqrt{4r^2+1}-1)^2}{4r^2},
		\end{equation}
причем для всех \(r \in R\).

		\begin{figure}[!h]
			\centering
			\includegraphics[width=\picsize]{\picpath ds2_stp}
			\caption{Положение неподвижной точки системы \eqref{ds2} в зависимостие от параметра.} 
		\end{figure}

 		Для разрешения вопроса об устойчивости неподвижной точки \eqref{stp_ds2} нам понадобится
		\begin{theorem}[\cite{DSMB}]\label{th1}
			Пусть задана динамическая система с дискретным временем \eqref{dds}:
			\[u \mapsto f(u), \quad u \in \Rn{n},\]
		где \(f\) ~--- гладкое отображение из \(\Rn{n}\) в \(\Rn{n}\). Предположим, что отображение имеет неподвижную точку \(u^* : f(u^*) = u^*\). Тогда неподвижная точка \(u^*\) устойчива (асимптотически), если все собственные значения \(\mu_1,\ldots,\mu_n\) матрицы Якоби вектор-функции \(f(u)\), вычисленной в точке \(u^*\), удовлетворяют условию \(\abs{\mu_i} < 1\). Если же хоть одно собственное значение удовлетворяет условию \(\abs{\mu_i} > 1\), то положение равновесия \(u^*\) неустойчиво.
		\end{theorem}
	
		Итак, рассмотрим матрицу Якоби \(J(u) = (j_{ik})\) функции \(f(u_1,u_2)\), задающей нашу систему в записи \eqref{ds2subst}, в точке \(u^*\).
		\begin{equation*}
			\newcommand*\dfdu[2]{\frac{\partial f_{#1}}{\partial u_{#2}}}
			\vat{J(u)}{u^*} = \vat{\begin{pmatrix} \dfdu{1}{1} & \dfdu{1}{2} \\[5pt] \dfdu{2}{1} & \dfdu{2}{2} \end{pmatrix}}{u^*} = \vat{\begin{pmatrix} \frac{r(1-u_2)}{2\sqrt{u_1}} & -r\sqrt{u_1} \\[5pt] 1 & 0 \end{pmatrix}}{u^*} = \begin{pmatrix} \frac{r(1-u^*)}{2\sqrt{u^*}} & -r\sqrt{u^*} \\[5pt] 1 & 0 \end{pmatrix} 
		\end{equation*}
		Упростим элементы \(j_{11}\) и \(j_{12}\).
		\begin{multline*}
			j_{11} = \frac{r(1-u^*)}{2\sqrt{u^*}} = \frac{r\left(1-\frac{(\sqrt{4r^2+1}-1)^2}{4r^2}\right)}{2\sqrt{\frac{(\sqrt{4r^2+1}-1)^2}{4r^2}}} = \frac{r\frac{4r^2-(\sqrt{4r^2+1}-1)^2}{4r^2}}{2\frac{\sqrt{4r^2+1}-1}{2r}} = \\ = \frac{\frac{4r^2-(\sqrt{4r^2+1}-1)^2}{4r}}{\frac{\sqrt{4r^2+1}-1}{r}} = \frac{4r^2-(\sqrt{4r^2+1}-1)^2}{4(\sqrt{4r^2+1}-1)} = \frac{4r^2-(4r^2+1) + 2\sqrt{4r^2+1} - 1}{4(\sqrt{4r^2+1}-1)} = \\ =  \frac{2(\sqrt{4r^2+1} - 1)}{4(\sqrt{4r^2+1}-1)} = \frac{1}{2}, \quad \text{ при } r \in R.
		\end{multline*}
			\[j_{12} =  -r\sqrt{u^*} = -r\frac{(\sqrt{4r^2+1}-1)^2}{4r^2} = -\frac{(\sqrt{4r^2+1}-1)^2}{4r}, \quad \text{ при } r \in R.\]

		Таким образом
			\[J(u^*) = \begin{pmatrix} \frac{1}{2} & -\frac{(\sqrt{4r^2+1}-1)^2}{4r} \\[5pt] 1 & 0 \end{pmatrix} \]
		Собственные значения вычисляются символьно и равны
			\[\mu_{1,2} = \frac{1\mp \sqrt{1-\frac{4(\sqrt{4r^2+1}-1)^2}{r}}}{4}.\]
		Графический анализ зависимости  \(\mu_1\) и \(\mu_2\) от параметра (см. рис. \ref{ds2_mu12_re}--\ref{ds2_mu12_abs}) показывает, что начиная с некоторого \(\hat r\) собственные значения становятся чисто мнимыми и комплексно сопряжёнными\footnote{что так же, очевидно, следует из того, что элементы матрицы \(J(u^*)\) суть вещественные числа, для любых допустимых значений параметра \(r\).}.

		\begin{figure}[!h]
			\centering
			\subfloat[\centering \(\mu_1\)] 
			{{\includegraphics[width=\subpicsize]{\picpath ds2_mu1_re.pdf} }}
			\qquad
			\subfloat[\centering мод2 \(\mu_2\)]
			{{\includegraphics[width=\subpicsize]{\picpath ds2_mu2_re.pdf} }}
			\caption{Действительная часть собственных значений.} \label{ds2_mu12_re}
		\end{figure}

		\begin{figure}[!h]
			\centering
			\subfloat[\centering \(\mu_1\)] 
			{{\includegraphics[width=\subpicsize]{\picpath ds2_mu1_im.pdf} }}
			\qquad
			\subfloat[\centering \(\mu_2\)]
			{{\includegraphics[width=\subpicsize]{\picpath ds2_mu2_im.pdf} }}
			\caption{Мнимая часть собственных значений.} \label{ds2_mu12_im}
		\end{figure}

При этом 
	\[\begin{cases}
		&\abs{\mu_{1,2}} < 1, \text{ при } r \in \inter{0}{r_0};\\
		&\abs{\mu_{1,2}} = 1, \text{ при } r = r_0;\\
		&\abs{\mu_{1,2}} > 1, \text{ при } r \in \inter{r_0}{r_{max}}.
	\end{cases}\]
Граничное значение \(r_0\) вычисляется численно и составляет приблизительно \(1.755\).

		\begin{figure}[!h]
			\centering
			\subfloat[\centering \(\mu_1\)] 
			{{\includegraphics[width=\subpicsize]{\picpath ds2_mu1_abs.pdf} }}
			\qquad
			\subfloat[\centering \(\mu_2\)]
			{{\includegraphics[width=\subpicsize]{\picpath ds2_mu2_abs.pdf} }}
			\caption{Модуль собственных значений.} \label{ds2_mu12_abs}
		\end{figure}


		Таким образом, пользуясь теоремой \ref{th1}, мы можем заключить, что система \eqref{ds2} при всех \(r \in R\) имеет единственную неподвижную точку \eqref{stp_ds2}, устойчивую при значениях \(r\) из интервала \(\inter{0}{r_0}\), и неустойчивую при \(r\) из \(\inter{r_0}{r_{max}}\).

	\subsection{Инвариантная кривая и бифуркация Неймарка--Сакера I}
		
		Казалось бы, изученное в предыдущем пункте поведение неподвижной точки \(u^*\) говорит о том, что в системе \eqref{ds2} происходит \emph{бифуркация Неймарка-Сакера}. Об этом свидетельствует

		\begin{definition}[\cite{DSMB}]
			Рассмотрим двумерную динамическую систему с дискретным временем
			\begin{equation}\label{dds2}
				u \mapsto f(u,\alpha),\: u = (u_1,u_2) \in \Rn{2}, \: \alpha \in \Rn{}.
			\end{equation}

			Бифуркация положения равновесия в системе \eqref{dds2}, соответствующая появлению мультипликаторов \(\abs{\mu_1} = \abs{\mu_2} = 1,\: \mu_1 = \overline{\mu_2}\), называется \emph{бифуркацией Неймарка-Сакера} или \emph{дискретной бифуркацией Хопфа}.
		\end{definition}

		Однако, как было отмечено непосредственно перед п. \ref{section_stp}, мы допускали, возможно, менее строгие, чем того требует система \eqref{ds2}, ограничения на параметр \(r\).

	\subsection{Снова об области допустимых параметров}

		Действительно, рассмотрим \(f(u_1,u_2) =  r \sqrt{u_1} (1-u_2)  = r\,g(u_1,u_2) \) ~--- правую часть первого уравнения системы \eqref{ds2subst}. Графический анализ (см. рис. \ref{pic_ds2_g}) функции \(g(u_1,u_2)\) показывает, что уже при \(r > 0\) её значения могут превышать единицу, вследствие чего переменные покидают фазовое пространство\footnote{Может показаться, что этим явлением можно пренебречь, и, например, естественным образом расширить фазовое пространство. Однако, в случае данной системы, подобное поведение приводит не только к выходу из \(U=\inter{0}{1}\times\inter{0}{1}\), но и вообще из \(\Rn{2}\).}. Таким образом, область допустимых параметров \(R\) системы \eqref{ds2} суть есть \(\inter{0}{1}\).

		\begin{figure}[!h]
			\centering
			\includegraphics[width=\bigpicsize]{\picpath ds2_g.png}
			\caption{\(g(u_1,u_2) = \sqrt{u_1} (1-u_2)\).}\label{pic_ds2_g} 
		\end{figure}

	\newpage

		Так, поскольку предполагаемая точка бифуркации \(r_0\) не входит в \(R\), \emph{никакой бифуркации Неймарка--Сакера в системе \eqref{ds2}, строго говоря, не происходит}. Фазовый портрет системы на \(R\) не меняется и имеет всего одну особенность ~--- точку устойчивого равновесия \(u^*\). 

	\bigskip

		Ввиду вышесказанного, следующий пункт носит скорее иллюстративный характер.
		
	\subsection{Инвариантная кривая и бифуркация Неймарка--Сакера II}			

		Допустим, что рассматриваемая система может быть корректно задана в области параметра \(r\), где должна происходить бифуркация например посредством сужения её фазового пространства до множества точек \(U' \subset U\), орбиты которых не покидают \(U\)\footnote{Если такое множество вообще непусто, устойчиво к малым изменениям параметра около точки бифуркации и т.д. Впрочем исследование этого вопроса явно выходит за рамки данной работы.}. В таком, случае мы получим, вообще говоря, другую систему. В частности, точка бифуркации \(r_0\) может изменится, что, судя по результатам компьютерного моделирования, и происходит. Ниже приведены графики\footnote{возможность построения которых, говорит о том, что \(u_0\), вероятно, принадлежит \(U'\).} орбиты точки \(u_0 = (0.7 \: 0.7)^T\), при различных значениях параметра \(r\).

		Полученные изображения дают основания заключить, что в "<скорректированной"> системе наблюдается суперкритическая\footnote{Подробнее см. \cite{DSMB} п. 7.3, п. 9.6.} бифуркация Неймарка--Сакера ~--- устойчивая неподвижная точка сменяется устойчивой замкнутой инвариантной кривой.

		\begin{figure}[!h]
			\centering
			\subfloat[\centering Устойчивая точка равновесия. \(r = r_0 - 0.37\).] 
			{{\includegraphics[width=\subpicsize]{\picpath ds2_NS1} }}
			\qquad
			\subfloat[\centering Рождение инвариантной кривой. \(r = r_0 - 0.341\).]
			{{\includegraphics[width=\subpicsize]{\picpath ds2_NS2} }}
			\caption{Потеря устойчивости.}
		\end{figure}

		\begin{figure}[!h]
			\centering
			\subfloat[\centering \(r = r_0 - 0.34\).] 
			{{\includegraphics[width=\subpicsize]{\picpath ds2_NS3} }}
			\qquad
			\subfloat[\centering \(r = r_0 - 0.33\).]
			{{\includegraphics[width=\subpicsize]{\picpath ds2_NS4} }}
			\caption{Рост инвариантной кривой.}
		\end{figure}

		\begin{figure}[!h]
			\centering
			\subfloat[\centering \(r = r_0 - 0.29\).] 
			{{\includegraphics[width=\subpicsize]{\picpath ds2_NS5} }}
			\qquad
			\subfloat[\centering  \(r = r_0 - 0.281\).]
			{{\includegraphics[width=\subpicsize]{\picpath ds2_NS6} }}
			\caption{Диссипация(?) инвариантной кривой.}
		\end{figure}

\newpage
	\begin{thebibliography}{0}
		\bibitem{DSMB} Братусь~А.~С., Новожилов~А.~С., Платонов~А.~П.
			\emph{Динамические системы и модели биологии}. М.: ФИЗМАТЛИТ, 2010.
		\bibitem{KURS} Алимов~Д.~А. кафедральный курс 
			\emph{Динамические системы и биоматематика}, 2021.
	\end{thebibliography}

\end{document}