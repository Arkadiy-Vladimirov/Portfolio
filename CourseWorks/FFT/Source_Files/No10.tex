\documentclass[oneside,final,11pt]{article}
\usepackage[utf8]{inputenc}
\usepackage[russianb]{babel}
\usepackage{vmargin}
\usepackage{amsmath}
\usepackage{amsfonts}
\usepackage{amssymb}
\usepackage[matrix,arrow,curve]{xy}
\usepackage{graphicx}
\setpapersize{A4}
\setmarginsrb{2cm}{1.5cm}{1cm}{1.5cm}{0pt}{0mm}{0pt}{13mm}
\usepackage{indentfirst}
\sloppy

\DeclareMathOperator{\sinc}{sinc}
\newcommand\abs[1]	{|#1|}
\newcommand\segm[2]{[#1,#2]}
\newcommand\lm{\lambda}
\newcommand\comm[1]{\bigl\{\text{#1}\bigr\}}

\newcommand\updwninfint[1]{\int\limits_{-\infty}^{\infty} #1 \, dt}
\newcommand\ft[1]{\updwninfint{#1 e^{-i \lm t}}}
\newcommand\upinfint[1]{\int\limits_{0}^{\infty} #1 \, dt}
\newcommand\dwninfint[1]{\int\limits_{-\infty}^{0} #1 \, dt}

\renewcommand{\le}{\leqslant}
\renewcommand{\ge}{\geqslant}

\newtheorem{statement}{Утверждение}

\begin{document}
	\begin{titlepage}
		\begin{centering}
			\includegraphics[width=0.5\textwidth]{msu.png}\\
			{\scshape Московский государственный университет имени М.~В.~Ломоносова}\\
			Факультет вычислительной математики и кибернетики\\
			Кафедра системного анализа\\
			\vfill
			{\LARGE Лабораторная работа №3}\\
			\vspace{1cm}
			{\Huge\bfseries "<Аппроксимация преобразования Фурье с помощью БПФ в среде MatLab">\\}
		\end{centering}
		\vspace{1cm}
		\begin{flushright}
			\begin{large}
				{\itshape Студент 315 группы\\}
				А.~А.~Владимиров\\
				\vspace{5mm}
				{\itshape Руководитель Практикума\\}
				к.ф.-м.н., доцент П.~А.~Точилин\\
			\end{large}
		\end{flushright}
		\vfill
		\begin{centering}
			Москва, 2021\\ 
		\end{centering}
	\newpage
	\end{titlepage}

	\setcounter{page}{2}
	\section{Постановка задачи}
		Требуется получить численную аппроксимацию преобразования Фурье
		\begin{equation} \begin{aligned} \label{ft}
			&F(\lambda) = \int\limits_{-\infty}^{\infty} f(t)e^{-i\lambda t}\, dt,\\ 
			&f(t) = \frac{1}{2\pi} \int\limits_{-\infty}^{\infty} F(\lambda)e^{i\lambda t}\, d\lambda 
		\end{aligned} \end{equation} 
		для набора функций
		\begin{equation} \begin{aligned} \label{functions}
			&f_1(t) = te^{-2\abs{t}}\sh(t),\\ 
			&f_2(t) = \frac{t}{2+2t+t^2},\\
			&f_3(t) = e^{-t^6}\arctg(t^2),\\
			&f_4(t) =	\left\{ \begin{aligned}
							\arctg(t^3),&\quad \abs{t+2} \le 1,\\
							0,& \text{ иначе.}
						\end{aligned}\right.
		\end{aligned} \end{equation} 
		Заданы частота дискретизации \(\Delta_t\) и окно \(\segm{a}{b}\), значения функции \(f_n, \; n = \overline{1,4}\)
		на котором мы можем использовать для нахождения приближения. \par
		Для фунций \(f_1(t)\) и \( f_2(t)\) результат требуется сравнить с аналитическим решением,
		предварительно посчитанным вручную.

	\section{Алгоритм программной реализации численного преобразования Фурье}
		Несмотря на тот факт, что в среде {\ttfamily MatLab} имеется специальная функция {\ttfamily fft} реализующая
		быстрое преобразование Фурье, для получения корректного численного приближения приходится немного
		поработать с данными как до, так и после применения {\ttfamily fft}.  Об алгоритме этой работы и пойдет речь.
		\par

	\subsection{Предварительные соображения} \label{math_section_1}
		Пусть требуется найти дискретное преобразование Фурье функции \(f(t)\). Даны окно \(\segm{0}{T}\) 
		\footnote{В действительности под этим окном мы подразумеваем смещенное на \(\frac{\Delta_t}{2}\), т.е
		\(\segm{-\frac{\Delta_t}{2}}{T-\frac{\Delta_t}{2}}\).},
		количество отсчетов \(N\), величина окна \(T\) и частота дискретизации \(\Delta_t = \frac{T}{N}\).\par
		Фактически наша задача сводится к нахождению преобразования
		\begin{equation} \label{cf1} %complicated formula 1
		\tilde f (t) = (f \cdot d_{\Delta_t} \cdot h_T \ast d_{T})(t) \leftrightarrow 
		(\frac{1}{\Delta_t} F \ast d_{\frac{2\pi}{\Delta_t}} \ast \sinc ({\scriptstyle \frac{\lambda T}{2}})
		\cdot d_{\Delta_{\lambda}})(\lambda) = \tilde F (\lambda)
		 \end{equation}
		Обозначим
		\[ \begin{aligned}
		&\tilde F_n = \tilde F (\lambda_n), \quad \lambda_n = \frac{2\pi n}{T}, \; n = \overline{0, N-1} \\
		&f_k = f(k\Delta_t), \quad k = \overline{0, N-1}.
		\end{aligned} \]
		Известно 
		\footnote{Смысл и подробное обоснование формул \eqref{cf1}, \eqref{cf2} см. \cite{PLF}.}
		, что
		\begin{equation} \label{cf2} %complicated formula 2
		\tilde F_n = \alpha_n, \quad \text{где } \alpha_n = \frac{2\pi}{T} \sum_{k=0}^{N-1} f_k e^{\frac{-2\pi ikn}{N}},
		\quad k = \overline{0, N-1}.
		\end{equation}
		Отсюда получим значения приближенного преобразования Фурье \( \hat F(\lambda)\) функции \(f(t)\) в точках
		\( \lambda_n \)
		\begin{equation} \label{rcf} %resulting complicated formula
		\hat F(\lm_n) = \hat F_n = \Delta_t \tilde F_n  = \frac{T}{N} \alpha_n =\frac{2\pi}{N}
		\sum_{k=0}^{N-1} f_k e^{\frac{-2\pi ikn}{N}} = \bigl\{  \text{обозначим сумму как \( \tilde\alpha_n \)} \bigr\} =
		\frac{2\pi}{N} \tilde\alpha_n.
		\end{equation}

	\subsection{Программная реализация в случае окна вида \(\segm{0}{T}\)} \label{prog_section_1}
		Как уже упоминалось, в среде {\ttfamily MatLab} реализована функция {\ttfamily fft}, с функциональной точки
		зрения в точности совпадающая с формулой
		\[ \tilde \alpha_n = \sum_{k=0}^{N-1} f_k e^{\frac{-2\pi ikn}{N}}. \]
		Но тогда, раз верно \eqref{rcf}, то задав сетку \verb !t_k = 0:dt:T;!
		(предварительно инициализировав переменную {\ttfamily dt} значением \( \Delta_t \) ) и сеточную функцию
		 \verb !f0_k = f(t_k);!,
		мы можем найти коэффициенты \( \tilde\alpha_n \)
		\begin{verbatim}  a_n = fft(f_k); \end{verbatim}
		и сразу же вычислить требуемую аппроксимацию \( \hat F(\lm) \)
		\begin{verbatim}  F0_n = 2*pi/N * a_n; \end{verbatim}
 		в точках \( \lm_n = \frac{2\pi n}{T} \) \quad (\verb ! l_n = 0:(2*pi/T):(2*pi*(N-1)/T); !).
		
		Мотивация именования переменных \verb ! f0_k ! и \verb ! F0_k ! будет понятная далее.

	\subsection{Еще немного математики}
		Пусть теперь требуется выбрать отсчеты функции \(f(t)\) из окна \(\segm{a}{b}\). Определим функцию
		\(\mathring f(t)  = f(t-a)\) отсчеты которой из окна \(\segm{0}{b-a}\) соответствуют требуемым отсчетам
		функции \(f(t)\). Вспомним одно из свойств преобразования Фурье, а именно
		\[ f(t-a) \leftrightarrow e^{-i\lm a} F(\lm). \]
		Но тогда
		\[ \xymatrix{
		f(t-a)  \ar@{=}[d] \ar@{<->}[r] & e^{-i\lm a} F(\lm) \ar@{=}[d] \\
		\mathring f(t) \ar@{<->}[r] & \mathring F(\lm). \\
		} \]
		А значит
		\begin{equation} \label{FTshift} 
			F(\lm) = e^{i\lm a} \mathring F(\lm). 
		\end{equation}
		Из пункта \ref{math_section_1} нам известна аппроксимация функции \(\mathring F(\lm) \) в точках \( \lm_n \)
		\[ \mathring F(\lm_n)  \approx \hat F(\lm_n) = \hat F_n \quad\lambda_n = \frac{2\pi n}{T}, \; n = \overline{0, N-1} \]
		Из чего, с учетом \eqref{FTshift}, делаем вывод
		\begin{equation} \label{approx}
			F(\lm_n) \approx e^{i\lm_n a} \hat F_n \quad\lambda_n = \frac{2\pi n}{T}, \; n = \overline{0, N-1} 
		\end{equation}
	\subsection{Программная реализация в случае окна вида \(\segm{a}{b}\)}
		В силу утверждения \eqref{approx} случай с общим видом окна \segm{a}{b} легко сводится к уже рассмотренному
		\ref{prog_section_1}. С небольшими добавлениями относительно пункта \ref{prog_section_1}
		приводим листинг итоговой программы
		\footnote{Считаем что переменные dt, N, a, b, T предварительно проинициализированны корректными значениями
		\( \Delta_t, N, a, b, T \) соответственно.
		}. \par
		\begin{verbatim}
			a0 = 0; b0 = b-a;             %для удобства явно зададим границы сдвинутого окна
			tn = a0:dt:(b0-dt/2);         %сетка \t_n
			dl = 2*pi/T;                  %шаг \Delta_lambda
			ln = 0:dl:2*pi*(N-1)/T;       %сетка \lambda_n
			f0 = @(t) fHandle(t-a);       %функция f_0 = f(t-a)
			f0n = f0(tn);                 %сеточная функция f_0n
			F0n = 2*pi/N * fft(f0n);      %образ Фурье сеточной функции f_0n
			Fn = exp(1i*ln*a).*F0n;       %образ Фурье сеточной функции f_0 (явно не заданной!)
		\end{verbatim}
		
	\section{Аналитический вывод преобразования Фурье некоторых функций}
		Для дальнейшей иллюстрации результатов работы программы будет полезно вручную вычислить преобразование
		Фурье некоторых функций. Благодаря этому, позже мы сможем наглядно сравнить численный результат с
		аналитическим. \par
		{\large {\sffamily Пример 1.} Функция \( f(t)= te^{-2\abs{t}}\sh(t). \) \\}
			Преобразование Фурье функции \(f(t)\) по определению имеет вид
			\[ F(\lm) = \ft{te^{-2\abs{t}}\sh(t)}. \]
			Воспользуемся свойством преобразования Фурье
			\[ (-it) f(t) \leftrightarrow F'(\lm): \]
			\[ F(\lm) = \ft{ (-it) i e^{-2\abs{t}}\sh(t) }, \text{ обозначим } g(t) = i e^{-2\abs{t}}\sh(t), \text{ тогда}\]
			\begin{equation} \label{F_eq_dG} 
			F(\lm) = \ft{(-it)g(t)} = G'(\lm), \text{ где } G(\lm) = \ft{g(t)}. \end{equation}
			Перейдем вычислению \(G(\lm)\)
			\begin{multline*} 
			G(\lm) = \ft{g(t)} = \comm{т.к \(g(t)\) ~--- нечетная} = i\updwninfint{g(t)\sin(\lm t)} = \\
			= \comm{раскроем модуль в экспоненте}  = i\dwninfint{i e^{2t} \sh(t) \sin(\lm t)} + 
			i\upinfint{i e^{-2t} \sh(t) \sin(\lm t)} = \\ =   -\dwninfint{e^{2t} \sh(t) \sin(\lm t)}
			-\upinfint{e^{-2t} \sh(t) \sin(\lm t)} = -I_1 - I_2.
			\end{multline*} 
			\[ I_1 = \comm{произведем замену \(x = -t\)} = \int_{\infty}^{0} e^{-2x}\sh(-x)\sin(-\lm x) \, (-dx) = 
			\int_{0}^{\infty} e^{-2x}\sh(x)\sin(\lm x) \, dx = I_2. \]
			Таким образом
			\[ G(\lm) = -I_1 - I_2 = - 2 I_2 = -2\upinfint{e^{-2t} \sh(t) \sin(\lm t)}, \]
			далее
			\begin{multline*}
			 G(\lm) = -2\upinfint{e^{-2t} \sh(t) \sin(\lm t)} = -2\upinfint{e^{-2t} \frac{e^t - e^{-t}}{2}
			\frac{e^{i\lm t} - e^{-i\lm t}}{2i}} =\\= \frac{i}{2}\upinfint{(e^{-t}- e^{-3t}) (e^{i\lm t} - e^{-i\lm t})} =
			\frac{i}{2}\upinfint{e^{(-1+i\lm)t} - e^{(-1-i\lm)t} - e^{(-3+i\lm)t} + e^{(-3-i\lm)t}} = \\ = 
			\frac{i}{2} \Bigl(\frac{1}{-1+i\lm} e^{(-1+i\lm)t} - \frac{1}{-1-i\lm}e^{(-1-i\lm)t} - 
			\frac{1}{-3+i\lm} e^{(-3+i\lm)t} + \frac{1}{-3-i\lm} e^{(-3-i\lm)t}\Bigr)\biggm|_{0}^{\infty} = \\
			=\frac{i}{2} \Bigl(\frac{1}{1-i\lm} - \frac{1}{1+i\lm} - \frac{1}{3-i\lm}  + \frac{1}{3+i\lm}\Bigr) = 
			\frac{i}{2} \Bigl( \frac{2i\lm}{1+\lm^2} - \frac{2i\lm}{9 +\lm^2}\Bigr) = \\
			= \Bigl( - \frac{1}{1+\lm^2} + \frac{1}{9 +\lm^2}\Bigr) = \frac{-8}{(1+\lm^2)(9+\lm^2)}.
			\end{multline*}
			Получив \(G(\lm)\), мы можем вернуться к вычислению \(F(\lm)\), которое, в силу \eqref{F_eq_dG}
			сводится к нахождению производной
			\[ \Bigl(\frac{-8}{(1+\lm^2)(9+\lm^2)}\Bigr)' = \Bigl(\frac{-8}{(9+10\lm^2 + \lm^4)}\Bigr)' = 
			\frac{8(20\lm+4\lm^3)}{(9+10\lm^2+\lm^4)^2} = \frac{32\lm(5+\lm^2)}{(1+\lm^2)^2(9+\lm^2)^2} \]
			Отдельно выпишем полученный результат
			\begin{equation} \label{F_1}
			F(\lm) =  \frac{32\lm(5+\lm^2)}{(1+\lm^2)^2(9+\lm^2)^2}. \end{equation}
			
		{\large {\sffamily Пример 2.} Функция \( f(t) = \frac{t}{2+2t+t^2}. \) \\ }
			Преобразование Фурье данной функции по определению имеет вид
			\begin{equation} \label{ex2} F(\lm) = \ft{\frac{t}{2+2t+t^2}}. \end{equation}
			Предварительно напомним важное следствие леммы Жордана \footnote{Также см. \cite{PLF}}.
			\begin{statement} \label{st1}
				Пусть \(f(z)\) аналитична при \(\abs{z} > R\) при достаточно большом \(R\), не имеет особых точек
			на \({\mathbb R }\) и \(f(z) \xrightarrow[\substack{\abs{z} \rightarrow 0 \\ im\,z \ge 0}]{} 0\),
			тогда при \(\lm > 0 \)
			\[ \int\limits_{-\infty}^{\infty} e^{i\lm x} f(x) \, dx = 
			2\pi i \sum\limits_{z_i\,:\,im\,z >0}^{} res(e^{i\lm z}f(z),z_i).\]
			\end{statement}
			Отметим, что при \(\lm \ne 0 \) мы находимся (с некоторыми поправками) в условиях утв.\ref{st1}.\par
			Перед вычислением интеграла \eqref{ex2} найдем особые точки подинтегральной функциии и
			соответствующие им вычеты.
			\[g(t) = \frac{t e^{-i\lm x}}{2+2t+t^2} = \frac{t e^{-i\lm x}}{(t+1)^2+1} = 
			\frac{t e^{-i\lm x}}{(t+1 - i)(t+1+i)}.\]
			Так, имеем два полюса \(z_1, z_2\) первого порядка с ненулевой мнимой частью
			\[ \begin{aligned}
				&z_1 = -1+i, \text{ находится в верхней полуплоскости,}\\
				&z_2 = -1-i, \text{ находится в нижней полуплоскости.}
			\end{aligned} \]
			Вычеты для них
			\[ \begin{aligned}
				&res(g(z),z_1) = \lim\limits_{z\to z_1} g(z)(z-z_1) = \frac{z_1 e^{-i\lm z_1}}{z_1 - z_2} = 
			\frac{(-1+i) e^{-i\lm (-1+i)}}{(-1+i) - (-1-i)} =  \frac{-1+i}{2i} e^{\lm (1+i)},\\
				&res(g(z),z_2) = \lim\limits_{z\to z_2} g(z)(z-z_2) = \frac{z_2 e^{-i\lm z_2}}{z_2 - z_1} =
			\frac{(-1-i) e^{-i\lm (-1-i)}}{(-1-i) - (-1+i)}	= \frac{1+i}{2i} e^{\lm (-1+i)}.
			\end{aligned} \]
			\par
			Теперь мы без труда можем найти интеграл \eqref{ex2}. Рассмотрим три случая
			\begin{description}
			\item[\(\lm > 0 :\)]
				\begin{multline*} F(\lm) = \ft{\frac{t}{2+2t+t^2}} = \comm{используем лемму Жордана для нижней
				полуплоскости} =\\
				= -2\pi i\; res(g(z),z_2) = -2\pi i \frac{1+i}{2i} e^{\lm (-1+i)} = \pi (-1-i) e^{\lm (-1+i)}.
				\end{multline*}
			\item[\(\lm < 0 :\)]
				\begin{multline*} F(\lm) = \ft{\frac{t}{2+2t+t^2}} = \comm{исп. л. Ж. для верхней полуплоскости} = \\
				= 2\pi i\; res(g(z),z_1) = 2\pi i \frac{-1+i}{2i} e^{\lm (1+i)} = \pi (-1+i) e^{\lm(1+i)}.
				\end{multline*}
			\item[\(\lm = 0 :\)]
				\begin{multline*}
				F(\lm) = \updwninfint{\frac{t}{2+2t+t^2}} =
				\int\limits_{-\infty}^{\infty} \frac{(t+1-1)\,dt}{(t+1)^2 + 1} =
				\frac{1}{2} \int\limits_{-\infty}^{\infty} \frac{2(t+1)\,dt}{(t+1)^2 + 1} -
				\int\limits_{-\infty}^{\infty} \frac{dt}{(t+1)^2+1} = \\
				= \frac{1}{2} \int\limits_{-\infty}^{\infty} \frac{dx}{x+1} - \int\limits_{-\infty}^{\infty} \frac{dy}{y^2+1}
				\stackrel{\text{(v.p)}}{=} \ln \abs{x+1}\biggm|_{-\infty}^{+\infty} -
				\arctg(y)\biggm|_{-\infty}^{+\infty} \stackrel{\text{(v.p)}}{=} 0 - \Bigl(\frac{\pi}{2} - (-\frac{\pi}{2})\Bigr)
				= -\pi.
				\end{multline*}
			\end{description}
			Итого
			\begin{equation} \label{F_2}
			F(\lm) = \left\{\begin{aligned}
				\pi (-1+i) e^{\lm(1+i)},&\quad \lm < 0, \\
				-\pi,&\quad \lm = 0, \\
				\pi (-1-i) e^{\lm (-1+i)},&\quad \lm >0. \end{aligned}\right.
			\end{equation}

	\section{Иллюстрации}
		Перейдем к результатам работы программы
	
	\begin{thebibliography}{0}
		\bibitem{PLF} Точилин~П.~А. кафедральный курс \emph{"<Преобразование Лапласа-Фурье">}, 2020.
	\end{thebibliography}
\end{document}